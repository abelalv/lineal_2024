\documentclass{article}  % Define el tipo de documento (article, report, book, etc.)
\usepackage[utf8]{inputenc}  % Permite usar caracteres especiales
\usepackage{amsmath}  % Paquete para mejorar las capacidades matemáticas
\usepackage{amssymb}  % Paquete para símbolos matemáticos adicionales

\title{Título del Documento}  % Título del documento
\author{Autor/a}  % Autor/a del documento
\date{\today}  % Fecha; puedes poner una fecha específica o usar \today para la fecha actual

\begin{document}

\maketitle  % Genera el título





\section*{Problema 1}

En una región en desarrollo, un estudio de planificación social y de infraestructura, realizado por el departamento de ingeniería civil, ha determinado que la ocupación de los habitantes afecta directamente las necesidades de infraestructura y servicios públicos. Según este estudio, la ocupación de un niño, cuando alcance la adultez, depende de la ocupación de su padre, y las probabilidades de transición están representadas en la siguiente matriz de transición.

Las ocupaciones consideradas son:
\begin{itemize}
    \item \textbf{P} = Profesional (requiere infraestructura avanzada, como oficinas y centros educativos).
    \item \textbf{F} = Agricultor (requiere zonas rurales y acceso a tierras agrícolas).
    \item \textbf{L} = Obrero (necesita zonas urbanas básicas, áreas de residencia industrial).
    \item \textbf{C} = Comerciante (requiere mercados y centros comerciales).
    \item \textbf{T} = Tecnólogo (necesita centros de innovación y tecnología).
\end{itemize}

La matriz de transición es la siguiente:

\[
P = 
\begin{bmatrix}
0.6 & 0.1 & 0.1 & 0.1 & 0.1 \\
0.2 & 0.5 & 0.2 & 0.1 & 0.1 \\
0.1 & 0.2 & 0.4 & 0.2 & 0.2 \\
0.05 & 0.1 & 0.2 & 0.5 & 0.15 \\
0.05 & 0.1 & 0.1 & 0.1 & 0.45 \\
\end{bmatrix}
\]

Cada elemento \( a_{ij} \) representa la probabilidad de que un hijo de una persona con ocupación \( j \) tenga la ocupación \( i \) en su adultez. Por ejemplo, la probabilidad de que el hijo de un profesional también sea profesional es \(0.6\), mientras que la probabilidad de que el hijo de un tecnólogo también se dedique a la tecnología es \(0.45\).

Este estudio es importante para prever las necesidades futuras de infraestructura en función de las ocupaciones, de modo que la región pueda planificar adecuadamente.

\section*{Preguntas}

\begin{enumerate}
    \item ¿Cuál es la probabilidad de que el nieto de un profesional también sea un profesional?

    Esta pregunta es clave para analizar si la ocupación profesional será dominante en futuras generaciones, lo cual ayuda a decidir si se debe invertir en infraestructura avanzada, como universidades y oficinas.

    \item ¿Cuál es la probabilidad de que el nieto de un tecnólogo también se dedique a la tecnología?

    Esto indicará si la tecnología seguirá siendo una ocupación relevante, permitiendo prever la necesidad de centros de innovación tecnológica.

    \item A largo plazo, ¿qué proporción de la población se dedicará a la agricultura?

    Esta información es esencial para la planificación de zonas rurales y para satisfacer la demanda de alimentos locales.

    \item ¿Cuál será la proporción a largo plazo de comerciantes en la población?

    Saber esta proporción ayuda a anticipar la necesidad de infraestructura comercial, como mercados y áreas de venta.

    \item ¿Qué proporción de la población estará dedicada a trabajos como obreros en el largo plazo?

    Este dato es relevante para planificar zonas residenciales y servicios básicos en áreas industriales.
\end{enumerate}

\newpage

\section*{Resolución}

\begin{enumerate}
    \item \textbf{Probabilidad de que el nieto de un profesional también sea un profesional:}
    
    Para obtener esta probabilidad, necesitamos calcular la probabilidad de transición en dos generaciones, es decir, encontrar el elemento \((P^2)_{11}\), donde \( P \) es la matriz de transición.

    \item \textbf{Probabilidad de que el nieto de un tecnólogo también sea tecnólogo:}
    
    Necesitamos calcular el elemento \((P^2)_{55}\) de la matriz \(P^2\), que nos da la probabilidad de que un tecnólogo tenga un nieto tecnólogo.

    \item \textbf{Proporciones a largo plazo de la población en cada ocupación:}
    
    Para encontrar la proporción a largo plazo en cada ocupación, debemos hallar el \textit{vector de estado estacionario} de la matriz \( P \), denotado como \( \pi \), que cumple con:
    
    \[
    \pi \cdot P = \pi
    \]
    
    y
    
    \[
    \pi_1 + \pi_2 + \pi_3 + \pi_4 + \pi_5 = 1
    \]

    Cada componente de \( \pi \) representa la proporción de la población en cada ocupación a largo plazo. 
\end{enumerate}

Con esta información, los ingenieros civiles y planificadores podrán anticipar las necesidades de infraestructura y servicios adecuados para satisfacer la evolución ocupacional de la población en el largo plazo.


\end{document}