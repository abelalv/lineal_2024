\documentclass{article}
\usepackage{amsmath}

\begin{document}

\title{Análisis de Vibraciones en un Sistema de Dos Grados de Libertad}
\author{}
\date{}
\maketitle

\section*{Problema}

Un ingeniero estructural está analizando un sistema de dos grados de libertad que representa un modelo simplificado de un edificio de dos pisos. Cada piso tiene una masa \( m \) y está conectado a un sistema de resortes que representa la rigidez estructural entre los pisos y el suelo. El objetivo es determinar las **frecuencias naturales de vibración** y los **modos de vibración** del sistema para evaluar el comportamiento del edificio bajo vibraciones.

\subsection*{Características del Sistema}

\begin{itemize}
    \item Cada piso (1 y 2) tiene una masa \( m \).
    \item La rigidez estructural (constante del resorte) entre el suelo y el primer piso es \( k_1 \).
    \item La rigidez estructural entre el primer y el segundo piso es \( k_2 \).
    \item Se supone que el sistema está sujeto a pequeñas oscilaciones alrededor de su posición de equilibrio.
\end{itemize}

\subsection*{Modelo Matemático}

Para este sistema, las frecuencias naturales y los modos de vibración se pueden determinar a partir de la **matriz de rigidez** \( K \) y la **matriz de masas** \( M \). Estas matrices están definidas como:

- La matriz de masas \( M \) es una matriz diagonal de tamaño \(2 \times 2\), dada por:
  \[
  M = \begin{bmatrix} m & 0 \\ 0 & m \end{bmatrix} = m I
  \]
  donde \( I \) es la matriz identidad.

- La matriz de rigidez \( K \) del sistema es:
  \[
  K = \begin{bmatrix} k_1 + k_2 & -k_2 \\ -k_2 & k_2 \end{bmatrix}
  \]

Para encontrar las frecuencias naturales y los modos de vibración, debemos resolver el problema de autovalores asociado a estas matrices.

\subsection*{Objetivo}

1. Determinar las **frecuencias naturales de vibración** del sistema, las cuales están relacionadas con los autovalores de la matriz de rigidez \( K \) en relación con la matriz de masas \( M \).
2. Encontrar los **modos de vibración** del sistema, que indican cómo se mueven los pisos (1 y 2) en cada frecuencia natural.

\section*{Procedimiento}

1. \textbf{Formulación del Problema de Autovalores}: Planteamos el problema de autovalores con la siguiente ecuación:
   \[
   \text{det}(K - \lambda M) = 0
   \]
   donde \(\lambda\) son los autovalores del sistema, relacionados con el **cuadrado de las frecuencias naturales**. Las frecuencias naturales \(\omega_i\) del sistema se calculan como:
   \[
   \omega_i = \sqrt{\lambda_i}
   \]
   donde \(\omega_i\) es la frecuencia natural correspondiente al autovalor \(\lambda_i\).

2. \textbf{Cálculo de Autovalores}:
   Expanda y resuelva el determinante para encontrar los autovalores \(\lambda_1\) y \(\lambda_2\).

3. \textbf{Cálculo de Autovectores}:
   Para cada autovalor \(\lambda_i\), calcule el autovector correspondiente. Estos autovectores describen los **modos de vibración**, es decir, las proporciones relativas de desplazamiento entre los dos pisos en cada frecuencia natural.

\section*{Preguntas}

\begin{enumerate}
    \item (a) Escriba el problema de autovalores para el sistema, es decir, encuentre \(\text{det}(K - \lambda M) = 0\).
    
    \item (b) Calcule los autovalores \(\lambda_1\) y \(\lambda_2\) del sistema, y luego determine las **frecuencias naturales** \(\omega_1\) y \(\omega_2\).

    \item (c) Calcule los autovectores correspondientes a \(\lambda_1\) y \(\lambda_2\). Estos autovectores representan los modos de vibración y muestran cómo se mueven los pisos en cada frecuencia natural.

    \item (d) Interprete los modos de vibración:
    \begin{itemize}
        \item ¿Cómo se desplaza el primer piso en comparación con el segundo en cada modo?
        \item ¿En qué dirección se desplazan los pisos para cada frecuencia natural?
    \end{itemize}

    \item (e) Respuesta del sistema a una excitación externa: Si el sistema experimenta una excitación externa con una frecuencia cercana a \(\omega_1\), ¿cómo responderá el sistema en términos de los modos de vibración?
\end{enumerate}

\section*{Solución Analítica}

\begin{enumerate}
    \item \textbf{Problema de Autovalores}

    La ecuación del problema de autovalores se obtiene planteando \(\text{det}(K - \lambda M) = 0\):
    \[
    K - \lambda M = \begin{bmatrix} k_1 + k_2 - \lambda m & -k_2 \\ -k_2 & k_2 - \lambda m \end{bmatrix}
    \]
    y resolvemos \(\text{det}(K - \lambda M) = 0\) para obtener los autovalores \(\lambda_1\) y \(\lambda_2\).

    \item \textbf{Cálculo de las Frecuencias Naturales}

    Las frecuencias naturales del sistema se obtienen como:
    \[
    \omega_1 = \sqrt{\lambda_1}, \quad \omega_2 = \sqrt{\lambda_2}
    \]

    \item \textbf{Cálculo de los Autovectores}

    Para cada autovalor \(\lambda_i\), el autovector correspondiente describe el modo de vibración del sistema. Resolvemos \((K - \lambda_i M) \mathbf{v}_i = 0\) para encontrar el autovector \(\mathbf{v}_i\), que representa las amplitudes relativas de desplazamiento entre los dos pisos en cada frecuencia natural.

    \item \textbf{Interpretación de los Modos de Vibración}

    Los autovectores \(\mathbf{v}_1\) y \(\mathbf{v}_2\) describen cómo se mueven los pisos en cada modo de vibración. Si ambos componentes del autovector tienen el mismo signo, los pisos se moverán en la misma dirección; si tienen signos opuestos, se moverán en direcciones opuestas.

    \item \textbf{Respuesta a una Excitación en la Frecuencia Natural}

    Si el sistema es sometido a una excitación externa con una frecuencia cercana a \(\omega_1\), el sistema mostrará un fenómeno de \textbf{resonancia} en el modo de vibración correspondiente a \(\lambda_1\), y los pisos vibrarán de acuerdo con el autovector \(\mathbf{v}_1\).
\end{enumerate}

\section*{Explicación del Problema}

Este problema permite entender las **frecuencias naturales** y los **modos de vibración** de un sistema estructural simple usando autovalores y autovectores. Estos resultados son cruciales para diseñar estructuras seguras y evitar fenómenos de resonancia que puedan comprometer la integridad de la estructura.

\end{document}
