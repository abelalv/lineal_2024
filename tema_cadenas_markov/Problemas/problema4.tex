\documentclass{article}
\usepackage{amsmath}

\begin{document}

\title{Análisis de Selección de Mezclas de Cemento en Proyectos de Construcción}
\author{}
\date{}
\maketitle

\section*{Problema}

Un equipo de ingenieros de materiales está evaluando distintas combinaciones de mezclas de cemento para optimizar la resistencia y durabilidad de las estructuras de construcción. A partir de pruebas en diversos proyectos, se han identificado cinco tipos principales de mezclas, cada una con diferentes probabilidades de ser reutilizada o cambiada en función de su desempeño.

Cada mezcla tiene propiedades específicas que la hacen adecuada para diferentes tipos de proyectos, y los ingenieros analizan la probabilidad de transición de una mezcla a otra dependiendo de la eficacia observada en proyectos previos.

Las mezclas principales consideradas son:
\begin{itemize}
    \item \textbf{Mezcla Alta Resistencia (A)}: Se utiliza principalmente en estructuras de carga, como columnas y muros de carga.
    \item \textbf{Mezcla Resistente a la Compresión (B)}: Ideal para pavimentos y losas de concreto expuestas a compresión.
    \item \textbf{Mezcla Impermeable (C)}: Usada en construcciones que requieren protección contra la humedad, como cimientos y sótanos.
    \item \textbf{Mezcla de Secado Rápido (D)}: Común en proyectos que requieren un fraguado rápido, como reparaciones de emergencia.
    \item \textbf{Mezcla Ecológica (E)}: Una mezcla que incorpora materiales reciclados y es ideal para proyectos sostenibles y ecológicos.
\end{itemize}

Las probabilidades de transición entre estas mezclas, al observar el desempeño de un proyecto terminado, son las siguientes:
\begin{itemize}
    \item La \textbf{Mezcla Alta Resistencia} tiene un 80\% de probabilidad de ser elegida nuevamente para un proyecto similar.
    \item La \textbf{Mezcla Resistente a la Compresión} tiene un 50\% de probabilidad de ser reutilizada y un 30\% de ser reemplazada por la \textbf{Mezcla Alta Resistencia} en proyectos con mayores cargas.
    \item La \textbf{Mezcla Impermeable} tiene un 60\% de probabilidad de mantenerse en proyectos de protección contra la humedad y un 20\% de ser sustituida por la \textbf{Mezcla de Secado Rápido} en proyectos de emergencia.
    \item La \textbf{Mezcla de Secado Rápido} tiene un 70\% de probabilidad de ser usada nuevamente en proyectos de fraguado rápido y un 10\% de probabilidad de cambiar a la \textbf{Mezcla Alta Resistencia} si se necesitan propiedades de soporte.
    \item La \textbf{Mezcla Ecológica} tiene un 90\% de probabilidad de mantenerse en proyectos sostenibles, y un 5\% de ser sustituida por la \textbf{Mezcla Resistente a la Compresión} en caso de requerir mayor resistencia estructural.
\end{itemize}

Estas probabilidades están representadas en la matriz de transición \( P \), que describe la probabilidad de elegir la misma mezcla o una diferente en el próximo proyecto.

\section*{Preguntas}

\begin{enumerate}
    \item (a) Escriba la matriz de transición para este proceso de Markov.
    
    \item (b) Si al inicio de un nuevo proyecto el 40\% de las mezclas seleccionadas son de \textbf{Mezcla Alta Resistencia}, ¿qué porcentaje de las mezclas se espera que sean de \textbf{Mezcla Alta Resistencia} en el siguiente proyecto?

    \item (c) Si se utilizan las mezclas en tres proyectos sucesivos, ¿cuál es la probabilidad de que la \textbf{Mezcla Resistente a la Compresión} se utilice en el tercer proyecto?

    \item (d) ¿Cuál es la probabilidad de que, después de dos proyectos, se utilice la \textbf{Mezcla Impermeable} en un proyecto de cimientos o sótanos?

    \item (e) A largo plazo, ¿qué proporción de los proyectos utilizará cada tipo de mezcla?

    \item (f) Si cada proyecto tiene un costo de 10,000 USD y los proyectos que usan \textbf{Mezcla Alta Resistencia} generan un 20\% más de ingresos, ¿cuál sería el ingreso esperado a largo plazo basado en el vector de estado estacionario?

    \item (g) ¿Qué proporción de los proyectos que comenzaron con \textbf{Mezcla de Secado Rápido} se espera que cambien a \textbf{Mezcla Alta Resistencia} después de cuatro proyectos?

    \item (h) Si cada cambio de mezcla tiene un costo adicional de 1,000 USD, ¿cuál es el costo esperado de los cambios en el largo plazo?

    \item (i) ¿Cuántos proyectos serían necesarios para que al menos el 50\% de los proyectos que comenzaron con la \textbf{Mezcla Impermeable} terminen utilizando la \textbf{Mezcla Alta Resistencia}?
\end{enumerate}


\end{document}
