\documentclass{article}
\usepackage{amsmath}

\begin{document}

\title{Análisis de Conversión de Clientes en un Proceso de Marketing}
\author{}
\date{}
\maketitle

\section*{Problema}

El departamento de marketing de una empresa de servicios envía periódicamente cartas promocionales a una extensa lista de clientes potenciales para invitarlos a suscribirse a un nuevo servicio de membresía premium. La lista de destinatarios incluye personas en distintos estados de compromiso con la empresa. Dependiendo de su interés y de la recepción de la carta, los clientes pueden moverse entre diferentes categorías, según el nivel de conversión que muestran:

\begin{itemize}
    \item **Clientes Activos (A)**: Ya están suscritos y tienden a renovar su membresía.
    \item **Clientes de Alto Interés (B)**: Han mostrado interés previo, como visitas al sitio web o consultas.
    \item **Clientes de Interés Medio (C)**: Son receptivos a las promociones, aunque su probabilidad de suscribirse es moderada.
    \item **Clientes de Bajo Interés (D)**: Muestran poco interés en las promociones, aunque es posible que una parte de ellos se suscriba.
    \item **Clientes Sin Interés (E)**: No responden a las campañas promocionales y tienen baja probabilidad de conversión.
\end{itemize}

Las transiciones entre estos estados de interés tras el envío de una carta se definen según la probabilidad de que los clientes de cada categoría pasen a otra o mantengan su estado actual:

\begin{itemize}
    \item Los **Clientes Activos** tienen un 80\% de probabilidad de renovar su membresía.
    \item Los **Clientes de Alto Interés** tienen un 50\% de probabilidad de pasar a ser **Clientes Activos** y suscribirse, y un 20\% de quedarse en el mismo estado.
    \item Los **Clientes de Interés Medio** tienen un 30\% de probabilidad de convertirse en **Clientes de Alto Interés** y un 10\% de pasar directamente a **Clientes Activos**.
    \item Los **Clientes de Bajo Interés** tienen un 5\% de probabilidad de pasar a ser **Clientes de Interés Medio** y un 5\% de suscribirse.
    \item Los **Clientes Sin Interés** tienen una probabilidad muy baja (1\%) de responder y suscribirse, permaneciendo en su mayoría sin cambios.
\end{itemize}

Estas probabilidades están resumidas en la matriz de transición \( P \), que refleja la probabilidad de cada cliente de permanecer en su estado actual o moverse a otro estado tras recibir una carta promocional.

\section*{Preguntas}

\begin{enumerate}
    \item (a) Escriba la matriz de transición para este proceso de Markov.
    
    \item (b) Si en la última campaña, el 40\% de los destinatarios ordenaron una suscripción, ¿qué porcentaje de las personas que reciben la carta actual se espera que pidan una suscripción?

    \item (c) Si se inicia una campaña mensual de envío de cartas, ¿cuál es la probabilidad de que un **Cliente de Interés Medio** (C) se convierta en **Cliente Activo** (A) en el tercer mes?

    \item (d) ¿Cuál es la probabilidad de que un **Cliente Sin Interés** (E) termine en el estado de **Cliente Activo** (A) tras dos campañas consecutivas?

    \item (e) A largo plazo, ¿qué proporción de la base de datos de clientes estará en cada estado?

    \item (f) Suponga que cada cliente genera un ingreso de 100 USD cuando se convierte en **Cliente Activo**. Si la empresa tiene 10,000 clientes en su base de datos, ¿cuál sería el ingreso esperado en el largo plazo basado en el vector de estado estacionario?

    \item (g) ¿Qué proporción de los **Clientes de Bajo Interés** (D) se espera que pasen al estado de **Cliente Activo** (A) después de cuatro campañas?

    \item (h) Si el costo de enviar una carta es de 2 USD por cliente, ¿cuál es el costo-beneficio esperado a largo plazo de la campaña de marketing?

    \item (i) ¿Cuántas campañas serían necesarias para que al menos el 50\% de los **Clientes de Interés Medio** (C) se conviertan en **Clientes Activos** (A)?
\end{enumerate}

\end{document}
